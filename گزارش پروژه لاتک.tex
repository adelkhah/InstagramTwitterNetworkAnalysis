\documentclass[12pt]{article}
\usepackage{amsmath}
\usepackage{amsfonts}
\usepackage{amssymb}
\usepackage{mathptmx}
\usepackage{geometry}
\geometry{left=0.75in, right = 0.75in,top=1.2in}
\usepackage{graphicx}
\usepackage{zref-perpage}
\usepackage{hyperref}
\usepackage{caption}
\usepackage[titles]{tocloft}
\usepackage{float}
\usepackage{enumerate}
\usepackage{listings}
\usepackage{xcolor}
\lstset{language=C++,
    commentstyle=\color{green},
    keywordstyle=\color{blue},
    numberstyle=\color{magenta}\tiny,
    stringstyle=\color{red},
    basicstyle=\ttfamily\footnotesize,
    breakatwhitespace=false,         
    breaklines=true,                 
    captionpos=b,                    
    keepspaces=true,                 
    numbersep=5pt,                  
    showspaces=false,                
    showstringspaces=false,
    showtabs=false,                  
    tabsize=4
}
\lstdefinestyle{customc}{
  belowcaptionskip=1\baselineskip,
  breaklines=true,
  frame=L,
  xleftmargin=\parindent,
  language=C,
  showstringspaces=false,
  basicstyle=\footnotesize\ttfamily,
  keywordstyle=\bfseries\color{green!40!black},
  commentstyle=\itshape\color{purple!40!black},
  identifierstyle=\color{blue},
  stringstyle=\color{orange},
}
\lstset{escapechar=@,style=customc}
\usepackage{fancyhdr}
%\usepackage[margin=1in]{geometry}
\usepackage{xepersian}
\settextfont[Scale = 1.2]{B Nazanin}
\setlatintextfont[Scale = 1.1]{Times New Roman}

\title{کاوش شبکه های مجازی اینستاگرام و توییتر به همراه تحلیل آنها }
\author{
استاد : دکتر قائمی بافقی
\\
دانشجو : علی عادلخواه
}


\date{تابستان 1402}
\begin{document}
\begin{center}
\includegraphics[scale=1.3]{thename.jpg}
\newpage
\includegraphics[scale=1.3]{ferdo.png}
\end{center}
\newpage

\tableofcontents

\newpage

\listoffigures

\newpage


\maketitle

\abstract{
در این گزارش شبکه های مجازی اینستاگرام و توییتر با استفاده از کتابخانه سلنیوم کاوش شده است. همچنین مطالعات و تحقیق درباره روش های داده کاوی و تحلیل فضای مجازی نیز صورت گرفته و تعدادی از روش ها مانند خوشه بندی، مسئله بیشترین تاثیر گذاری و پیشبینی پیوند ها با کمک هوش مصنوعی پیاده سازی شده است.
}

\section{مقدمه}
\subsection{تاریخچه}
در سال 1967 توسط آقای
\lr{stanley milgraml}
 نظریه ای مطرح شد كه هر كسي به طور متوسط با ۶ قدم ميتواند با تمام دنيا در ارتباط باشد.
\cite{smallworld}

وي همچنين تحقيقي با استفاده از تقريبا ۳۰۰ نفر داوطلب در مورد اين قضيه انجام داد و مقاله اي در سال 
1977 منتشر كرد.
\cite{swp}


\subsection{مقالات مرتبط و کار های انجام شده}
در سال 
2007
آقای 
\lr{Lescovec}  
  تحقیقی بر روي گرافي متشكل از ۲۴۰ ميليون راس كه متعلق به شبکه
\lr{MSN Messenger }  
    بود انجام داد و ميانگين فاصله بين راس ها را ۶ .۶ بدست آورد.
\cite{msn}


و اما فيسبوك در سال ۲۰۱۲ تحقيقی انجام داد بر روي ۷۲۱ ميليون كاربري كه در آن زمان داشت و 
همانطور كه در نمودار زير مشاهده ميكنيد بيشترين فاصله بين دو راس ۱۰ بوده است و ميانگين فاصله 4.74
كه نشان ميدهد به طور متوسط از هر جاي جهان تقريبا با ۵ قدم ميتوان با تمام مردم جهان ارتباط برقرار 
كرد.
\cite{fourdeg}

\begin{figure}[htbp]
\centering
\includegraphics[scale=0.8]{facebook.jpg}
\caption{توزیع فاصله در گراف فیسبوک 2012}
\end{figure}

\newpage

در سال ۲۰۱۶ فيسبوك بار ديگر بر روي ديتاست كاربران خود كه آن زمان $1.6$ ميليارد جمعيت داشته است 
ميانگين فاصله را محاسبه میکند.
\cite{facebook}

\begin{figure}[htbp]
\centering
\includegraphics[scale=0.8]{face.jpg}
\caption{توزیع فاصله در گراف فیسبوک سال 2016}
\end{figure}

\newpage


در سال ۲۰۱۳ نيز در مقاله ای برای توييتر و ديتاستي از ۴۷۰ ميليون كاربر ميانگين فاصله افراد محاسبه شده است. 
نتيجه را در نمودار زير با مقايسه با سال 2009 مي بينيد.
\cite{degseptwitter}

\begin{figure}[htbp]
\centering
\includegraphics[scale=0.8]{twitterpath.jpg}
\caption{توزیع فاصله در توییتر سال 2013}
\end{figure}

\subsection{کاربرد های تحلیل فضای مجازی}
از كاربرد هاي كاوش رسانه هاي ديجيتال در زمينه اقتصادي و سياسي ميتوان به موارد زير اشاره كرد.
\cite{ecpo}


\lr{Market research}
:
 کسب و کار ها میتوانند با استفاده از تحلیل فضای مجازی و پیدا کردن سلایق افراد جامعه هدف خود را انتخاب کنند و تبلیغ موثر برای آنها داشته باشند.

\lr{Customer Service}
:
 از طریق فضای مجازی میتوان با مشتری در ارتباط بود و با تحلیل بازخورد میتوان محصولات را ارتقا بخشید.


\lr{Risk management}
:
 با نظارت بر فضای مجازی میتوان تهدید های پیشرو را پیشبینی کرد و از آنها جلوگیری کرد

\lr{Competitive intelligence}
:
 تحلیل فضای مجازی به شرکت ها این امکان را میدهد تا استراتژی های شرکت های رقیب خود را پیدا کنند و در رقابت با آنها از این اطلاعات برای بهتر کردن استراتژی های خود استفاده کنند.

\lr{Political communication}
:
 تحلیل فضای مجازی میتواند خبر های بسیار مهمی درباره اتفاقات سیاسی و بازخورد مردم به آنها را در اختیار تحلیل کنندگان بگذارد.



\lr{Election forecasting}
:
 با تحلیل فضای مجازی میتوان نتیجه انتخابات را پیشبینی کرد.

\lr{Political risk Management}
:
با تحلیل فضای مجازی میتوان تهدید های سیاسی را نیز پیشبینی کرد و از آنها جلوگیری کرد.



\subsection{نرم افزار های مرتبط}
نرم افزار های زیادی در رابطه با داده کاوی و کاوش شبکه های مجازی وجود دارد که شاید معروف ترین آنها 
\lr{gephi}
و
\lr{NodeXL}
باشد.
در این مقاله تعدادی از این نرم افزار ها بررسی و با هم مقایسه شده اند.
\cite{tools}
و خلاصه مقایسه آنها را در جدول زیر مشاهده میکنید.

\begin{figure}[htbp]
\centering
\includegraphics[scale=0.8]{toole.jpg}
\caption{مقایسه ابزارهای کاوش و تحلیل فضای مجازی}
\end{figure}


\section{مباحث نظری}
\subsection{گراف}
ارتباطات بین افراد در شبکه های مجازی را میتوان به راحتی به صورت یک گراف در نظر گرفت و الگوریتم های گراف و مباحث نظری گراف را در آن استفاده کرد.
اولین مبحث نظری که به آن میپردازیم معیار مرکزیت در گراف است.
\cite{center}

\subsection{معیار مرکزیت}

\begin{figure}[htbp]
\centering
\includegraphics[scale=0.8]{centers.png}
\caption{معیار های مرکزیت در گراف}
\end{figure}


\begin{enumerate}
\item
\lr{betweenness}
:
 بر اساس تعداد راه هايي كه بين نود ها از اين راس عبور ميكنند.

\[
g(v) = \sum_{s\neq v \neq t} \frac{\sigma_{st}(v)}{\sigma_{st}}
\]

\item
\lr{degree}
:
بر اساس تعداد همسايه هاي يك راس.

\item
\lr{pagerank}
:
یك روش بازگشتي است كه 
\lr{pagerank}
 يك نود از 
 \lr{pagerank}
  همسايه هاي آن بدست
مي آيد.

\[
PR(u) = \sum_{v \in B_u} \frac{PR(v)}{L(v)}
\]
\item
\lr{closeness}
:
بر اساس كمترين فاصله اي كه دور ترين راس دارد.


\end{enumerate}


\subsection{\lr{Finding Communities}}
\subsubsection{\lr{Modularity}}
براي خوشه بندی گراف ابتدا لازم است معياری برای بررسی ميزان خوب بودن يک خوشه بندی داشته 
باشيم معياری که برای اين کار استفاده ميشود 
\lr{modularity}
 نام دارد و به شكل زير تعريف ميشود.
 
\[
\sum_u \left(\frac{m_u}{m} - \left( \frac{k_u}{2m} \right) ^ 2 \right)
\]

كه در آن 
$m$
 تعداد يال هاي كل گراف است. و 
 $m_u$
  تعداد يال هاي داخل خوشه 
  $u$
   است. و 
   $k_u$
    مجموع 
درجات راسهاي خوشه 
$u$
 است.
 
\subsubsection{الگوریتم های خوشه بندی}
خوشه بندی گراف خود یک مسئله 
\lr{NP Hard}
است و الگوریتم های بسیاری برای خوشه بندی معرفی شده اند که در زير تعدادی از آنها به همراه 
پيچيدگی زمان آمده است.

\begin{figure}[htbp]
\centering
\includegraphics[scale=1.5]{cluster.png}
\caption{الگوریتم های خوشه بندی با پیچیدگی زمانی آنها}
\end{figure}

\newpage

در این مقاله 7 الگوریتم خوشه بندی بالا را در 4 دیتاست با یکدیگر مقایسه کرده است.
\cite{lee}
و نتایج زیر بدست آمده اند.

\begin{figure}[htbp]
\includegraphics[scale=0.3]{clustermod.png}
\includegraphics[scale=0.4]{clustertime.png}
\caption{مقایسه الگوریتم های خوشه بندی}
\end{figure}


\subsection{\lr{Influence Maximization}}
بیشترین تاثیر گذاری یا 
\lr{Influence Maximization}
نیز یک مسئله 
\lr{NP Hard}
است.
تعریف مسئله به صورت مقابل است : میخواهیم یک مجموعه از راس های گراف را انتخاب کنیم و آنها را فعال کنیم و بعد از مدتی که راس های دیگر فعال شدند بیشترین تعداد راس ممکن فعال شده باشد.

در سال ۲۰۰۳ در مقاله اي كه 
\lr{Kempe}
 منتشر كرد
 نشان داد كه مسئله 
\lr{Maximization Influence}
از
\lr{NP Hard}
 است و يك روش حريصانه براي حل آن پيشنهاد داد . 
 که در آن روش در هر مرحله راسی 
که بيشترين تاثير را بر روی راس های ديگر داشت انتخاب ميشد. همچنين ثابت کرد در شرايط خاص اين
الگوريتم حداقل 63 درصد از بيشترين تاثيرپذيری ايده آل را خواهد داشت.
 \cite{greedy}

ال ۲۰۰۷ در مقاله اي كه 
\lr{Leskovec}
 منتشر كرد الگوريتمي ارائه شد كه امروزه به الگوريتم 
 \lr{celf} 
معروف است. و دقت الگوريتم حريصانه را داشت اما در زمان بسياركمتري عمل ميكرد.
\cite{celf}

 در قسمت پياده 
سازی اين روش ها پياده سازی و با هم مقايسه شده اند و نمودار های آنها رسم شده اند.


\subsubsection{\lr{Linear Threshold}}
 در اين حالت يک راس زمانی فعال ميشود که تعداد مشخصی از همسايه 
های آن فعال شده باشند.


\subsubsection{\lr{Independent Cascade}}
در اين حالت به هر يال يک احتمال نسبت داده ميشود و اگر يک 
طرف اين يال فعال شود با احتمالی که مشخص شده است راس طرف ديگر يال نيز فعال ميشود.
در اين مسئله هدف پيدا کردن يک مجموعه با اندازه مشخص از راس هاست که در صورت فعال بودن آن 
راس ها بيشترين تعداد ممکن از راس ها نيز فعال شوند.


\subsection{\lr{Link Prediction}}

مسئله بعدی پيشبينی پيوند های جديد بين دو کاربر است که اين موضوع با پيدا کردن شباهت های دو 
کاربر انجام ميشود از معيار هايی که برای تشابه بين دو کاربر ميتوان در نظر گرفت تعداد همسايه های 
مشترک و تشابه متن ها و تصوير هايی که در اکانت خود به اشتراک گذاشتن است.
مورد اول را به راحتی ميتوان محاسبه کرد اما برای مورد دوم نياز است تا از هوش مصنوعی کمک 
گرفت تا ميزان تشابه بين متن ها و تصوير ها را پيدا کند با استفاده از اين قابليت حتی ميتوان متن ها و 
تصوير ها را در کتگوری ها مختلف دسته بندی کرد و بر اساس شباهتی که به هر کتگوری دارند در آن 
قرار گيرند زيرا يکی از تحليل های کاربردی در متن و شبکه های مجازی تحليل احساسی پيام ها است. 

\section{پیاده سازی}
\subsection{تجهیزات مورد نیاز}
قسمت فرانت اند این پروژه با
\lr{JavaScript}
و
\lr{CSS}
و
\lr{HTML}
پیاده سازی شده است. و قسمت بکند با زبان پایتون و کتابخانه
\lr{fastapi}
پیاده سازی شده است. 
تمام کتابخانه ها و توابع مورد استفاده را با توضیحات میتوانید در 
\lr{code documentation}
 مشاهده کنید.

\subsection{کاوش اینستاگرام و توییتر}
برای کاوش شبکه های مجازی از کتابخانه 
\lr{selenium}
و مرورگر گوگل کروم استفاده شده است.
\cite{sel}


\subsection{نمایش گراف ارتباطات}
گراف شبکه های مجازی گراف های بسیار حجیم و بزرگی هستند و روش های معمول برای نشان دادن گراف آنها به خوبی عمل نمیکند در این پروژه با استفاده از زبان جاوا اسکریپت و کتابخانه 
\lr{d3}
گراف ها نمایش داده شده اند.
\cite{d3}

\subsubsection{گراف اینستاگرام}
در مجموع 29 هزار کاربر اینستاگرام کاوش شده اند و گراف ارتباطات آنها را در زیر مشاهده میکنید.

\begin{figure}[htbp]
\centering
\includegraphics[scale=0.4]{insta.png}
\caption{گراف 29 هزار کاربر اینستاگرام}
\end{figure}


همانطور که میبینید تعداد ارتباطات در این گراف به غیر از مرکز گراف بسیار کمتر است.
علت آن نعداد زیاد اکانت های خصوصی در اینستاگرام است. (حدود 75 درصد)




\subsubsection{گراف توییتر}
در مقابل در توییتر تنها 5 درصد اکانت ها خصوصی بودند و بقیه اکانت ها قابل کاوش بودند در زیر گراف ارتباطات توییتر را 
میبینید که به نسبت گراف اینستاگرام بسیار متراکم تر است.

\begin{figure}[htbp]
\centering
\includegraphics[scale=0.4]{twitter.png}
\caption{گراف 22 هزار کاربر توییتر}
\end{figure}

و نکته قابل توجهی که در این گراف مشاهده میشود دو قطبی شدن افراد به دسته است.
دسته سمت راست بیشتر تگ های مربوط به اربعین استفاده شده است و میتوان نتیجه گرفت داخل این بخش بیشتر افراد مذهبی قرار دارند. و اما در طرف مقابل بیشتر تگ های مهسا امینی استفاده شده است و میتوان گفت این دسته بیشتر از افراد غیر مذهبی تشکیل شده است.

\subsection{نمایش گراف تگ ها}
گراف تگ ها ميتواند رابطه بين تگ ها را نشان دهد. در اين گراف هر تگ يک راس و اگر دو تگ در 
يک پيام با همديگر استفاده شده باشند بين آنها يال وجود خواهد داشت. همچنين تعداد استفاده از يک تگ 
ميتواند نشان دهنده اهميت آن موضوع باشد و با تغيير اندازه راس های گراف اين اهميت را به صورت
بصری نشان داد در زير گراف تگ های شبکه توييتر را مشاهده ميکنيد.

\begin{figure}[htbp]
\centering
\includegraphics[scale=0.4]{tag.png}
\caption{گراف تگ های توییتر}
\end{figure}

با يک نگاه به اين گراف ميتوان از اهميت يک تگ که اندازه بزرگ تری نسبت به بقيه دارد با خبر شد.


\subsection{الگوریتم خوشه بندی}
همانطور که در بخش مباحث نظری قسمت خوشه بندی توضیح داده شد الگوریتم 
\lr{louvain}
الگوریتم بسیار مناسبی برای خوشه بندی است زیرا هم دقت بسیار خوبی دارد و هم سرعت بسیار خوبی.
در این پروژه نیز برای خوشه بندی از الگوریتم 
\lr{louvain}
استفاده شده است.
راس هایی که رنگ یکسان دارند بدین معنی است که در یک خوشه قرار گرفته اند.
در زیر خوشه بندی برای گراف اینستاگرام را مشاهده میکنید.
\begin{figure}[htbp]
\centering
\includegraphics[scale=0.3]{insta cluster1.png}
\caption{خوشه بندی گراف اینستاگرام}
\end{figure}
\newpage


\subsection{الگوریتم های بیشترین تاثیر گذاری}

همانطور که در بخش مباحث نظری توضيح داده شد اين مسئله را ميتوان به دو صورت مدل سازی کرد که هر کدام 
کاربرد خودشان را دارند اما در شبکه های اجتماعی حالت 
\lr{Independent Cascade}
 کاربرد بيشتری دارد و به 
واقعيت نزديک تر است. در اين پروژه نيز از اين مدل سازی استفاده شده و احتمال انتشار برابر با 0.1 در 
نظر گرفته شده است.
در اين حالت چون برای اجرا های مختلف ممکن است نتيجه متفاوتی دريافت شود از اين رو معمول است 
که ميانگين چندين بار اجرا در نظر گرفته شود. همچنين بايد در نظر داشت اگر تعداد اجرا ها زياد باشد 
گرچه عدد نهايی به 
\lr{value expected }
نزديک تر است اما همچنين باعث کند شدن برنامه نيز ميشود. همچنين 
اگر اندازه مجموعه اوليه نيز زياد باشد به همان اندازه زمان نياز است تا افراد مناسب پيدا شوند از اين رو 
در اين پروژه برای داشتن سرعت و دقت کافی اندازه مجموعه اوليه 20 و تعداد اجرا 100 در نظر گرفته 
شده است. 

\subsubsection{مقایسه معیار های مرکزیت}

در ابتدا فرض کنيد بدون استفاده از الگوريتم هايی که در اين زمينه ارائه شده اند بخواهيم با معيار های
مرکزيت راس ها اين مسئله را حل کنيم در اين پروژه ابتدا دو معيار 
\lr{pageRank}
 و 
 \lr{degree}
  بررسی شده اند 
نتايج را بر روی گراف اينستاگرام در زير مشاهده ميکنيد.

\begin{figure}[htbp]
\includegraphics[scale=0.4]{degree pagerank influence.jpg}
\includegraphics[scale=0.4]{degree pagerank time.jpg}
\caption{
مقایسه دو معیار
\lr{pagerank}
و
\lr{degree}
در گراف اینستاگرام
}
\end{figure}
\newpage

از نظر زمانی دو الگوريتم بسيار سريع هستند.
نکته جالب توجه اين است که هر دو الگوريتم در ابتدا رشد بسيار سريعی دارند اما بعد از آن کند ميشوند. 
اين احتمال وجود دارد که به دليل ارتباط کم بين خوشه ها اين اتفاق افتاده است و انتشار از خوشه به سختی 
به خوشه های ديگر منتقل شده است. از اين رو در مرحله بعد سعی شده است در هر خوشه راس های
مرکزی بر حسب 
\lr{pageRank}
 و 
 \lr{degree}
انتخاب شوند و نتيجه به صورت زير شده است.

\begin{figure}[htbp]
\centering
\includegraphics[scale=0.4]{cluster degree pagerank.jpg}
\caption{
مقایسه دو معیار
\lr{pagerank}
و
\lr{degree}
در هر خوشه گراف اینستاگرام
}
\end{figure}


همچنين بررسی شده است اگر معيار مرکزيت را 
\lr{betweenness}
 در نظر بگيريم تا انتشار بين خوشه ای
بيشتر شود چه تاثيری بر روی انتشار خواهد گذاشت نتيجه را در زير مشاهده ميکنيد.


\begin{figure}[htbp]
\centering
\includegraphics[scale=0.4]{between and degree pagerank.jpg}
\caption{
مقایسه سه معیار
\lr{pagerank}
و
\lr{degree}
و
\lr{betweenness}
در گراف اینستاگرام
}
\end{figure}

\newpage
همچنين اين 3 معيار بر روی گراف توييتر نتيجه زير را به همراه داشته اند.
\begin{figure}[htbp]
\centering
\includegraphics[scale=0.4]{twitter pg bt deg.jpg}
\caption{
مقایسه سه معیار
\lr{pagerank}
و
\lr{degree}
و
\lr{betweenness}
در گراف توییتر
}
\end{figure}


\subsubsection{مقایسه الگوریتم ها}
دو الگوریتم حریصانه که در سال 2003 و الگوریتم 
\lr{celf}
که در سال 2007 
ارائه شده اند نیز پیاده سازی شده است و بر روی دیتاست اینستاگرام نتیجه زیر را به همراه داشتند.

\begin{figure}[htbp]
\includegraphics[scale=0.4]{greedy celf.jpg}
\includegraphics[scale=0.4]{greedy celf time.jpg}
\caption{مقایسه دو الگوریتم
\lr{greedy}
 و 
\lr{celf}
}
\end{figure}


همانطور که مشاهده ميکنيد در مقايسه با الگوريتم حريصانه که بعد از 5000 ثانيه به نتيجه رسيده است 
الگوريتم 
\lr{celf}
 با صرف زمان بسيار کمتر به همان نتيجه رسيده است.
همچنين با مجموعه اوليه 20 نفر اين الگوريتم ها انتشار را به 475 اکانت رسانده اند اما با استفاده از معيار 
های اکثريت تقريبا به 360 اکانت انتشار رسيده است.

\newpage

\subsection{کلمات پر تکرار در کنار تگ ها }
يک از روش های تحليل محتوای شبکه های مجازی مشاهده کلماتی است که در کنار هر تگ استفاده شده 
اند اين کار يک ايده کلی ميدهد که مردم در کنار آن تگ درباره چه چيز هايی حرف ميزنند و آن تگ چه 
معنايی برای مردم دارد. در زير کلمات پر تکرار و جفت کلمه های پر تکرار در کنار تگ های توييتر را 
مشاهده ميکنيد.

\begin{figure}[htbp]
\centering
\includegraphics[scale=0.25]{tagword.png}
\caption{کلمات پر تکرار در کنار تگ های توییتر}
\end{figure}

\newpage

\subsection{تحلیل احساسی پیام های هر تگ}
روش ديگر تحليل با استفاده از هوش مصنوعی آن است که پيام ها را بر اساس احساسی که منتقل ميکنند
طبقه بندی کرد. اين کار نيز يک ديد کلی ميدهد که افراد درباره هر تگ چه احساسی دارند.
نمونه ای از اين تحليل را در زير برای تگ های توييتر مشاهده ميکنيد.
هر پیام در یکی از دسته های 
\lr{joy, sad, hate, love, angry, fear, none}
قرار گرفته است.

\begin{figure}[htbp]
\centering
\includegraphics[scale=0.25]{tsent.png}
\caption{دسته بندی احساسی توییت ها هر تگ}
\end{figure}

\newpage

\subsection{پیشبینی پیوند های جدید}
برای اين قسمت ابتدا الزم است تعيين کنيم چه خصوصياتی باعث ميشود يک اکانت ديگر اکانت را دنبال 
کند.
اين اتفاق ميتواند به داليلی مانند محتوای مشترک يا تعداد دوستان يکسان و عاليق مشترک ... صورت 
پذيرد.
در اين پروژه دو معيار تعداد همسايه های يکسان و شباهت محتوای دو اکانت در نظر گرفته شده اند.

( در توييتر متن و در اينستاگرام تصوير)
مورد اول به راحتی قابل محاسبه است و برای مورد دوم با استفاده از هوش مصنوعی و تبديل متن و 
تصوير به بردار و اندازه گيری شباهت کسينوسی آنها ميتوان تا حدی شباهت عکس ها و متن ها پی برد.
برای داده های تصویری از مدل آماده کتابخانه 
\lr{imgbedding}
استفاده شده است.
\cite{ibed}

از آنجایی که کاربران هدف هم از زبان فارسی و انگلیسی استفاده میکنند.
دو مدل آماده هوش مصنوعی استفاده شده است.
برای زبان فارسی از مدل آماده  کتابخانه 
\lr{hazm}
\cite{hazm}
و برای زبان انگلیسی از مدل آماده کتابخانه 
\lr{sent2vec}
\cite{sentvec}
استفاده شده است.
همچنین از کتابخانه
\lr{langdetect}
\cite{detect}
برای تشخیص زبان استفاده شده است.
در زير پيش بينی پيوند برای کاربران توييتر را مشاهده ميکنيد.
\begin{figure}[htbp]
\centering
\includegraphics[scale=0.25]{instasim.png}
\caption{پیشبینی پیوند های جدید در اینستاگرام}
\end{figure}

\newpage

\section{نتیجه}

\subsection{نتایج پروژه}
در این پروژه کاوش شبکه های مجازی توییتر و اینستاگرام را به همراه تحلیل آنها بررسی کردیم.
دیدیم که در اینستاگرام تقریبا 75 درصد اکانت ها خصوصی بودند و برای کاوش و تحلیل شبکه مجازی توییتر که تقریبا 5 درصد اکانت ها خصوصی هستند بسیار مناسب تر است و تحلیل های بهتری میتوان انجام داد.
همچنین در گراف ارتباطات توییتر میتوان دو قطبی شدن جامعه ایران به دو دسته مذهبی و غیر مذهبی را کاملا مشاهده کرد.
و در گراف تگ ها دیدیم که بعد از یک سال هنوز هم از تگ مهسا امینی بسیار استفاده میشود.
و برای خوشه بندی گراف با استناد به مقاله ای که بررسی شد الگوریتم خوشه بندی
\lr{louvain}
نسبت به دیگر الگوریتم های خوشه بندی گفته شده بهتر و مناسب تر است.

همچنین در مسئله بیشترین تاثیر گذاری کاملا نشان داده شد که دقت الگوریتم
\lr{celf}
کاملا برابر است با دقت الگوریتم حریصانه با این تفاوت که بسیار سریع تر از آن الگوریتم عمل میکند.
و چنانچه برای مسئله بیشترین تاثیر گذاری قصد استفاده از الگوریتم هایی که در این زمینه ارائه شده اند را نداشته باشیم. و بخواهیم با استفاده از معیار های مرکزیت این مسئله را حل کنیم بهترین معیار 
\lr{betweenness}
خواهد بود و در رده های بعدی
\lr{degree}
و
\lr{pagerank}

\subsection{پروژه های پیشنهادی}
 از پروژه های دیگری که در این زمینه میتوان انجام داد به موارد زیر میتوان اشاره کرد.

\begin{enumerate}
\item
آموزش یک مدل هوش مصنوعی برای تحلیل احساسی پیام ها و شباهت متن و تصویر. در این پروژه از مدل های آماده کتابخانه های پایتون استفاده شد که شاید دقت کافی را نداشته باشند.


\item
کاوش بیشتر فضای مجازی توییتر و اینستاگرام. در این پروژه به دلیل محدودیت زمانی تنها 22 هزار کاربر توییتر و 29 هزار کاربر اینستاگرام کاوش شده اند که نسبت به تعداد زیر کاربران این شبکه های مجازی رقم کمی است.

\item
پیاده سازی دیگر الگوریتم های مسئله بیشترین تاثیر گذاری و مقایسه آنها در این پروژه تنها دو الگوریتم حریصانه و 
\lr{celf}
پیاده سازی و مقایسه شده اند.

\item
مطالعه و بررسی دیگر روش های داده کاوی در متن و تصاویر استخراج شده از فضای مجازی در این پروژه تنها تحلیل احساسی پیام ها با هوش مصنوعی و کلمات پر تکرار در کنار تگ ها و بررسی شباهت دو کاربر بر اساس محتوای آنها پیاده سازی شده است.

\end{enumerate}



\bibliographystyle{plain}
\begin{latin}
\bibliography{refs}
\end{latin}


\end{document}